%\documentclass[draft,hyperref={pdftex,pdfpagemode=none,pdfstartview=FitH}]{beamer}
\documentclass[hyperref={pdftex,pdfpagemode=none,pdfstartview=Fit}]{beamer}
\usepackage{mathptmx,helvet}
\usepackage[scriptsize]{subfigure}
\usepackage[3D]{movie15}
%\usepackage{media9}
%\usepackage{movie15}
\usepackage{xcolor,tikz}
\usepackage{animate}
\usetikzlibrary{arrows}

\newcommand{\norm}[1]{\ensuremath{\left\| #1 \right\|}}
\newcommand{\abs}[1]{\ensuremath{\left| #1 \right|}}
\newcommand{\bracket}[1]{\ensuremath{\left[ #1 \right]}}
\newcommand{\braces}[1]{\ensuremath{\left\{ #1 \right\}}}
\newcommand{\parenth}[1]{\ensuremath{\left( #1 \right)}}
\newcommand{\ip}[1]{\ensuremath{\langle #1 \rangle}}
\newcommand{\tr}[1]{\mbox{tr}\ensuremath{\!\bracket{#1}}}
\newcommand{\deriv}[2]{\ensuremath{\frac{\partial #1}{\partial #2}}}
\newcommand{\SO}{\ensuremath{\mathsf{SO(3)}}}
\newcommand{\T}{\ensuremath{\mathsf{T}}}
\newcommand{\G}{\ensuremath{\mathsf{G}}}
\renewcommand{\L}{\ensuremath{\mathsf{L}}}
\newcommand{\R}{\ensuremath{\mathsf{R}}}
\newcommand{\I}{\ensuremath{\mathsf{I}}}
\newcommand{\so}{\ensuremath{\mathfrak{so}(3)}}
\newcommand{\SE}{\ensuremath{\mathsf{SE(3)}}}
\newcommand{\se}{\ensuremath{\mathfrak{se}(3)}}
\renewcommand{\Re}{\ensuremath{\mathbb{R}}}
\newcommand{\Sph}{\ensuremath{\mathsf{S}}}
\newcommand{\aSE}[2]{\ensuremath{\begin{bmatrix}#1&#2\\0&1\end{bmatrix}}}
\newcommand{\ase}[2]{\ensuremath{\begin{bmatrix}#1&#2\\0&0\end{bmatrix}}}
\newcommand{\D}{\ensuremath{\mathbf{D}}}
\newcommand{\pair}[1]{\ensuremath{\left\langle #1 \right\rangle}}
\newcommand{\met}[1]{\ensuremath{\langle\!\langle #1 \rangle\!\rangle}}
\newcommand{\Ad}{\ensuremath{\mathrm{Ad}}}
\newcommand{\ad}{\ensuremath{\mathrm{ad}}}
\newcommand{\g}{\ensuremath{\mathfrak{g}}}
\newcommand{\intp}{\ensuremath{\mathbf{i}}}
\newcommand{\extd}{\ensuremath{\mathbf{d}}}
\newcommand{\hor}{\ensuremath{\mathrm{hor}}}
\newcommand{\ver}{\ensuremath{\mathrm{ver}}}
\newcommand{\dyn}{\ensuremath{\mathrm{dyn}}}
\newcommand{\geo}{\ensuremath{\mathrm{geo}}}

\definecolor{RoyalBlue}{rgb}{0.25,0.41,0.88}
\def\emph{\textcolor{RoyalBlue}}

\definecolor{mygray}{gray}{0.9}

\mode<presentation> {
  \usetheme{Warsaw}
  \usefonttheme{serif}
  \setbeamercovered{transparent}
}

\newcommand{\mypaper}{}
\newcommand{\myhead}{}

\setbeamertemplate{headline} {%
  \leavevmode%
  \hbox{\begin{beamercolorbox}[wd=.5\paperwidth,ht=2.5ex,dp=1.125ex,leftskip=.3cm,rightskip=.3cm]{author in head/foot}\hfill\insertsection
  \end{beamercolorbox}%
  \begin{beamercolorbox}[wd=.5\paperwidth,ht=2.5ex,dp=1.125ex,leftskip=.3cm,rightskip=.3cm]{title in head/foot}\vskip2pt\myhead\vskip1pt
  \end{beamercolorbox}}%
  \vskip0pt%
}


\setbeamertemplate{footline}%{split theme}
{%
  \leavevmode%
  \hbox{\begin{beamercolorbox}[wd=.5\paperwidth,ht=2.5ex,dp=1.125ex,leftskip=.3cm,rightskip=.3cm plus1fill]{author in head/foot}%
    \usebeamerfont{author in head/foot}\insertshorttitle
  \end{beamercolorbox}%
  \begin{beamercolorbox}[wd=.5\paperwidth,ht=2.5ex,dp=1.125ex,leftskip=.3cm,rightskip=.3cm]{title in head/foot}
%    \usebeamerfont{title in head/foot}\mypaper\hfill \insertframenumber/\inserttotalframenumber
    \usebeamerfont{title in head/foot}\hfill \insertframenumber/\inserttotalframenumber
  \end{beamercolorbox}}%
  \vskip0pt%
} \setbeamercolor{box}{fg=black,bg=yellow}

\title[Adaptive Plasma Technologies]{Adaptive and Controllable Plasma Technologies:\\ Toward Autonomous Cancer Treatments}

\author{$ $\\Taeyoung Lee, Michael Keidar\vspace*{0.1cm}\\
{\footnotesize\selectfont %Flight Dynamics and Control Laboratory\\
Mechanical and Aerospace Engineering\\ George Washington University}\\ $ $\\
%{\scriptsize\selectfont Collaboration with N. Harris McClamroch (UMich), Melvin Leok (UCSD),\\ Vijay Kumar (UPenn), Fred Leve, Brien Flewelling (AFRL),\\ Tse-Huai Wu, Farhad Goodarzi, Evan Kaufmann, Daewon Lee (GWU)}
}


\institute[]{\normalsize NSF IUCRC IAB Meeting\\ August 2018}%\\ $ $\\
%\institute[]{\small\selectfont March 2016\\ $ $\\
%{\scriptsize
%$^\star$Supported by AFRL, NSF, NRL, and ONR}}

\date{}

\definecolor{tmp}{rgb}{0.804,0.941,1.0}
\setbeamercolor{numerical}{fg=black,bg=tmp}
\setbeamercolor{exact}{fg=black,bg=red}

\newcommand{\mediafolder}{/Users/tylee/Documents/Seminar/Drexel18/Figs}
\newcommand{\mediafoldera}{/Users/tylee/Documents/Seminar/Drexel18/Figs/adaptive/ppt/media}
\newcommand{\mediafolderb}{/Users/tylee/Documents/Seminar/Drexel18/Figs/MPC/ppt/media}
\graphicspath{{\mediafolder/},{\mediafoldera/}}


%\includeonly{Sec5}

\begin{document}

\begin{frame}
  \titlepage
\end{frame}

\begin{frame}
    \frametitle{Adaptive and Controllable Plasma Technologies}

    \begin{itemize}
        \item Overview
            \begin{itemize}
                \item Plasma has been successfully applied to \emph{cancer therapy}
                \item Cell response to plasma depends on various treatment conditions and cancer type
                \item Develop \emph{foundation and principles} for plasma cancer therapy
                \item Construct \emph{optimal, feedback treatment strategies adaptive to cell response and diagnostics}, utilizing control system engineering
            \end{itemize}
    \end{itemize}
    \vspace*{0.3cm}

    \centerline{
        \includemovie[draft,autostart,loop,poster,repeat,attach=false]{0.25\textwidth}{0.25\textwidth}{\mediafoldera/media1.mov}
        \hspace*{0.3cm}
        \includegraphics[height=0.25\textwidth]{image7.jpeg}
        \hspace*{0.3cm}
        \includegraphics[height=0.25\textwidth]{image3.png}
    }

\end{frame}


\begin{frame}
    \frametitle{CAP Cancer Therapy}

    \begin{itemize}
        \item Cold Atmospheric Plasma (CAP)
            \begin{itemize}
                \item Ionized gases with ion temperature close to room temperature
                \item Exhibit unique chemical and physical properties, such as reactive species, charges, and electric fields
            \end{itemize}
    \end{itemize}
    \vspace*{0.1cm}

    \centerline{
        \includegraphics[height=0.35\textwidth]{Picture1.png}
        \hspace*{0.3cm}
        \includegraphics[height=0.4\textwidth]{CAP}
    }
\end{frame}

\begin{frame}
    \frametitle{CAP Cancer Therapy}

    \begin{itemize}
        \item CAP Cancer Treatment 
            \begin{itemize}
                \item CAP jet is applied to cancer cell or CAP treated medium is injected
                \item CAP provides a rich environment of reactive oxygen/nitrogen species (RONS), charged particles, photons and an electric field
                \item Assembling these species in various combination provides unprecedented, synergistic effects in cancer therapy
            \end{itemize}
            \vspace*{0.3cm}
        \item Treatment Mechanism
            \begin{itemize}
                \item Selective rise of intracellular ROS initiates ROB-based death pathways
                \item CAP causes severe damages in DNA, mitochondria, or cell membrane, which leads to apoptosis, i.e. \textit{programmed  death} of cancer cell
                \item Exact mechanism to be defined
            \end{itemize}
    \end{itemize}]
\end{frame}

\begin{frame}
    \frametitle{CAP Cancer Therapy}

    \begin{itemize}        
        \item Synergistic Effects
            \begin{itemize}
                \item Anticancer ability: inhibit cancer growth
                \item Selectivity: eradicate cancer cells while preserving normal cells
                \item Restoration: reinstate therapeutic effects of chemotherapy 
                \item Illustrated by various experiments in vitro and in vivo
            \end{itemize}
    \end{itemize}
    \vspace*{0.5cm}
    \centerline{
        \scriptsize\selectfont
        \hspace*{-0.3cm}
        \begin{tabular}{cc}
            Apoptosis (Kieft et al. 2004) &
            Growth inhibition  (Fridman et al. 2007) \\
            Cytoskeletal damage (Lee et al. 2009) &
            Selectivity  (Georgescu et al. 2010)\\
            Cell cycle arrest  (Kim et al. 2010) &
            DNA, mitochondrial damage (Kim et al. 2010) \\
            Growth inhibition in vivo (Vandamme et al. 2011)& 
            Increased intracellular ROS (Ahn et al. 2011) \\
            Selective increase of ROS (Kim et al. 2013) &
            Immunogenic cancer cell death (Lin et al. 2015) \\
            Cell-based $H_2O_2$ generation (Keidat et al. 2017) & $\cdots$ \\
        \end{tabular}
    }
\end{frame}


\begin{frame}
    \frametitle{CAP Cancer Therapy}
    \framesubtitle{Micro CAP Treatment for Glioblastoma in vitro}

    \centerline{
        \includegraphics[height=0.47\textwidth]{image20.png}
        \hspace*{0.5cm}
        \includegraphics[height=0.5\textwidth]{image19a}
    }

        \begin{center}
            \scriptsize\selectfont
            Z. Chen et al. `A Novel Micro Cold Atmospheric Plasma Device for Glioblastoma Both in Vitro and in Vivo'' \textit{Cancers} 9.6 (2017): 61.
    \end{center}
\end{frame}

\begin{frame}
    \frametitle{Adaptive Plasma Technologies} 

    \begin{itemize}
        \item Motivation
            \begin{itemize}
                \item Cancer cell response to CAP depends on various factors, including cancer type, CAP discharge voltage, gas flow rate and composition, and treatment duration 
            \end{itemize}
    \end{itemize}

    \centerline{
        \includegraphics[width=0.85\textwidth]{image28}
    }
\end{frame}

\begin{frame}
    \frametitle{Adaptive Plasma Technologies} 

    \begin{itemize}
        \item Adaptive Plasma Technologies
            \begin{itemize}
                \item Goal: develop feedback system based on cellular response and boundary conditions for plasma discharge
                \item Investigate characteristics of cancer cell response to CAP under varying treatment conditions
                \item Develop in situ diagnostics of cell response
                \item Construct feedback treatment strategies adaptive to cell response
            \end{itemize}
    \end{itemize}
    \centerline{
        \includegraphics[width=0.65\textwidth]{image29}
    }
\end{frame}

\begin{frame}
    \frametitle{Adaptive Plasma Technologies} 
    
    \begin{itemize}
        \item Interdisciplinary Research Toward Autonomous Cancer Therapy
            \begin{itemize}
                \item Interplay between plasma physics, mathematical oncology, control system engineering for the development of intelligent biomedical system
                \item Adaptivity: treatment tailored to the particular cancer cell response
                \item Optimality: treatment schedule optimized for cancer viability, selectivity, etc
                \item Robustness: reject undesired behavior caused by modeling errors and uncertainties
                \item Autonomy: autonomous scheduling for dose and frequency
            \end{itemize}
    \end{itemize}
\end{frame}

\begin{frame}
    \frametitle{Adaptive Plasma Technologies} 
    
    \begin{itemize}
        \item Feedback Control System
            \begin{itemize}
                \item Designed based on a mathematical description of a dynamic system
                \item Dynamic model can be represented either by
                    \begin{itemize}
                        \item differential equations constructed by first principles
                        \item a large set of data for input and output responses
                    \end{itemize}
            \end{itemize}
            \vspace*{0.3cm}
        \uncover<2->{
        \item Adaptive Plasma Technology
            \begin{itemize}
                \item Challenges: complex dynamic system with limited number of data
                \item Hybrid approach to construct an analytic model based on our knowledge and adjust it with experiments
            \end{itemize}
        }
    \end{itemize}
    \vspace*{0.3cm}
            \centerline{
        \scriptsize\selectfont
		\tikzset{>=latex}
        \begin{tikzpicture}
            \draw[latex-latex, line width=0.05cm, <->] (0,0) -- (7.5,0);
            \node[below] at (0,0) {\shortstack[c]{Mechanical and robotic\\systems}};
            \node[above] at (0,0) {\shortstack[c]{Analytic model\\derived by first principles\\$\dot x =f(t,x,u)$}};
            \node[below] at (7.5,0) {\shortstack[c]{Artificial intelligence\\Deep learning}};
            \node[above] at (7.5,0) {\shortstack[c]{Input-output data\\from experiments\\$\{x_i(t),u_i(t)\}_{i=1}^N$}};
        \only<2->{
            \draw[-] (4.5,-0.2) -- (4.8,0.2) node[above] {I. MPC of U87};
            \draw[-] (2.8,0.2) -- (2.8,-0.2) node[below] {\shortstack[c]{II. Oxidative stress\\based model}};
        }
        \end{tikzpicture}
    }
\end{frame}


\begin{frame}
    \frametitle{Model Predictive Control}

    \begin{itemize}
        \item Objective
            \begin{itemize}
                \item To adjust CAP treatment conditions adaptively based on the actual cancer cell response
                \item Develop Model Predictive Feedback Control (MPC) scheme for CAP treatment
            \end{itemize}
            \vspace*{0.3cm}
        \item Mathematical Modeling
            \begin{itemize}
                \item Cell viability for various treatments conditions is measured in vitro
                \item Mathematical model is developed to predict the cell viability
            \end{itemize}
            \vspace*{0.3cm}
        \item Optimal Control
            \begin{itemize}
                \item Optimal control problem is formulated to reduce the cancer viability below the desired level
                \item Optimal control is repeatedly applied based on the actual cancer cell response (feedback)
            \end{itemize}
    \end{itemize}
\end{frame}

\begin{frame}
    \frametitle{Model Predictive Control}

    \begin{itemize}
        \item Data Collection
            \begin{itemize}
                \item CAP is applied to U87 for varying treatment periods
                \item Cell viability is measured by real time-Glo cell viability assay over 48 hours
                \item Multiple temporal measurements indicate dynamic response of cancer cell to CAP
            \end{itemize}
    \end{itemize}
    \vspace*{0.3cm}
    \centerline{
        \includegraphics[height=0.3\textwidth]{\mediafolderb/image4}
        \hspace*{0.3cm}
        \includegraphics[height=0.3\textwidth]{\mediafolderb/image5}
    }
        \begin{center}
            \scriptsize\selectfont
            Gjika et al. ``Adaptation of operational parameters of cold atmospheric plasma for in vitro treatment of cancer cells'' ACS Appl. Mater. Interfaces 2018
    \end{center}
\end{frame}

\begin{frame}
    \frametitle{Model Predictive Control}
    \framesubtitle{Cell Viability for Varying Treatment Time}

    \only<1>{
        \centerline{
            \includegraphics[height=0.45\textwidth]{\mediafolderb/image8}
            \includegraphics[height=0.45\textwidth]{\mediafolderb/image9}
        }
        \centerline{$\Delta t = 0$\hspace*{0.5\textwidth}$\Delta t = 30\,\mathrm{sec}$}
    }

    \only<2>{
        \centerline{
            \includegraphics[height=0.45\textwidth]{\mediafolderb/image8}
            \includegraphics[height=0.45\textwidth]{\mediafolderb/image9}
        }
        \centerline{$\Delta t = 0$\hspace*{0.5\textwidth}$\Delta t = 30\,\mathrm{sec}$}
    }

\end{frame}


\end{document}
